\documentclass[paper=a4, fontsize=11pt]{scrartcl}
%\usepackage{sectsty} % Allows for custom section title styling
\usepackage[T1]{fontenc}
\usepackage{fourier} % We're using the Utopia font because it's great.
\usepackage{enumerate} % Allows for custom enumeration types
\usepackage{tikz} % Graphics powered by TikZ!
\usepackage{xcolor} % More colors
\usepackage{listings} % For shell output

\usetikzlibrary{automata} % Handy stuff for state machine diagrams

\lstdefinestyle{ShellStyle} {
  basicstyle=\small\ttfamily,
  numbers=none,
  frame=tblr,
  columns=fullflexible,
  backgroundcolor=\color{blue!10},
  linewidth=0.9\linewidth,
  xleftmargin=0.1\linewidth
}

%\allsectionsfont{\centering \normalfont\scshape} % Center and semi-cap section titles
\newcommand{\horrule}[1]{\rule{\linewidth}{#1}} % Horizontal rule with weight arg

\title{
  \normalfont \normalsize 
  \textsc{New Mexico Tech} \\ [25pt]
  \horrule{0.5pt} \\[0.4cm]
  \huge CSE 325 --- Lab Project 4 \\ Thread Scheduler \\
  \horrule{2pt} \\[0.5cm]
}

\author{Rob Kelly \& Ian Neal \\ SANIC TEEM}
\date{\normalsize\today}

\begin{document}
\maketitle

%%% INTRODUCTION %%%
The SANIC TEEM Amazing Thread Scheduling Demonstration for Peace (hereafter referred to as \textit{the program}) is an interactive simulation of user-mode thread scheduling, designed to demonstrate two different thread scheduling policies:

\begin{description}
  \item[First Come, First Served] or \textit{FIFO}, wherein processes are scheduled to run in the order they arrive in the scheduling queue and will continue running until they either complete or block for I/O, and

  \item[Round Robin] or \textit{RR}, wherein processes are granted CPU time in discrete time slices or \textit{quanta}, after which they are moved to the end of the scheduling queue if they have yet to finish running.
\end{description}

Most of the codebase of this project was given as part of the lab assignment. We have implemented the functionality of this project in the following files:

\begin{itemize}
  \item Interface and implementation of scheduling-policy-independent functionality in \texttt{sched\_impl.h} and \texttt{sched\_impl.c}, respectively.

  \item Interface and implementation of \textit{FIFO}-scheduling functionality in \texttt{fifo\_impl.h} and \texttt{fifo\_impl.c}, respectively.

  \item Interface and implementation of \textit{RR}-scheduling functionality in \texttt{rr\_impl.h} and \texttt{rr\_impl.c}, respectively.

  \item Build targets for the added files and this README in the \texttt{Makefile}.
\end{itemize}

%%% BUILDING %%%
\section*{Building}
\begin{itemize}
  \item \texttt{make} to build the project normally.

  \item \texttt{make test} to run the provided testing harness.

  \item \texttt{make doc} to build this README. Requires \texttt{pdflatex} and a number of \LaTeX\hspace{0em} packages, all of which are included in the popular \textbf{TeX Live} distribution.

  \item \texttt{make clean} to clean up temporary files, build files, and output.
\end{itemize} 

%%% USAGE %%%
% Taken from scheduler.c:print_help()
\section*{Usage}
\texttt{./scheduler <sched\_impl> <queue\_size> <num\_threads> [iterations]}

\begin{itemize}
  \item \texttt{sched\_impl} may be \texttt{-fifo} to use \textit{FIFO} scheduling, \texttt{-rr} to use \textit{RR} scheduling, or \texttt{-dummy} to offload thread scheduling to the kernel.

  \item \texttt{queue\_size} is the number of threads that can be in the scheduler at one time.

  \item \texttt{num\_threads} is the number of worker threads to run.

  \item \texttt{iterations} is optionally the number of loops for each worker thread to run.
\end{itemize}


%%% EXAMPLES %%%
% Taken from the assignment PDF. We could replace this with our own program's output at some point.
\section*{Examples}
This first example uses \textit{FIFO} scheduling:
\begin{lstlisting}[style=ShellStyle]$
  $ ./scheduler -fifo 1 2 3
  Main: running 2 workers on 1 queue_size for 3 iterations
  Main: detaching worker thread 3075984304
  Main: detaching worker thread 3065494448
  Main: waiting for scheduler 3086474160
  Thread 3075984304: in scheduler queue
  Thread 3075984304: loop 0
  Thread 3075984304: loop 1
  Thread 3075984304: loop 2
  Thread 3075984304: exiting
  Thread 3065494448: in scheduler queue
  Thread 3065494448: loop 0
  Thread 3065494448: loop 1
  Thread 3065494448: loop 2
  Thread 3065494448: exiting
  Scheduler: done!
\end{lstlisting}
\pagebreak
Another example, this time using \textit{RR} scheduling:
\begin{lstlisting}[style=ShellStyle]$
  $ ./scheduler -rr 10 2 3
  Main: running 2 workers on 10 queue_size for 3 iterations
  Main: detaching worker thread 3075828656
  Main: detaching worker thread 3065338800
  Main: waiting for scheduler 3086318512
  Thread 3075828656: in scheduler queue
  Thread 3065338800: in scheduler queue
  Thread 3075828656: loop 0
  Thread 3065338800: loop 0
  Thread 3075828656: loop 1
  Thread 3065338800: loop 1
  Thread 3075828656: loop 2
  Thread 3065338800: loop 2
  Thread 3075828656: exiting
  Thread 3065338800: exiting
  Scheduler: done!
\end{lstlisting}

%%% DESIGN %%%
\section*{Design \& Implementation}
% TODO

%%% BUGS %%%
\section*{Known Bugs}
% TODO

\end{document}
